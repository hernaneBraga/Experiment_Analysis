% Options for packages loaded elsewhere
\PassOptionsToPackage{unicode}{hyperref}
\PassOptionsToPackage{hyphens}{url}
%
\documentclass[
]{article}
\usepackage{lmodern}
\usepackage{amsmath}
\usepackage{ifxetex,ifluatex}
\ifnum 0\ifxetex 1\fi\ifluatex 1\fi=0 % if pdftex
  \usepackage[T1]{fontenc}
  \usepackage[utf8]{inputenc}
  \usepackage{textcomp} % provide euro and other symbols
  \usepackage{amssymb}
\else % if luatex or xetex
  \usepackage{unicode-math}
  \defaultfontfeatures{Scale=MatchLowercase}
  \defaultfontfeatures[\rmfamily]{Ligatures=TeX,Scale=1}
\fi
% Use upquote if available, for straight quotes in verbatim environments
\IfFileExists{upquote.sty}{\usepackage{upquote}}{}
\IfFileExists{microtype.sty}{% use microtype if available
  \usepackage[]{microtype}
  \UseMicrotypeSet[protrusion]{basicmath} % disable protrusion for tt fonts
}{}
\makeatletter
\@ifundefined{KOMAClassName}{% if non-KOMA class
  \IfFileExists{parskip.sty}{%
    \usepackage{parskip}
  }{% else
    \setlength{\parindent}{0pt}
    \setlength{\parskip}{6pt plus 2pt minus 1pt}}
}{% if KOMA class
  \KOMAoptions{parskip=half}}
\makeatother
\usepackage{xcolor}
\IfFileExists{xurl.sty}{\usepackage{xurl}}{} % add URL line breaks if available
\IfFileExists{bookmark.sty}{\usepackage{bookmark}}{\usepackage{hyperref}}
\hypersetup{
  pdftitle={Estudo de caso 03: Comparação de desempenho de duas configurações de um algoritmo de otimização},
  pdfauthor={Lucas Carneiro; Vinicius Ferreira},
  hidelinks,
  pdfcreator={LaTeX via pandoc}}
\urlstyle{same} % disable monospaced font for URLs
\usepackage[margin=1in]{geometry}
\usepackage{color}
\usepackage{fancyvrb}
\newcommand{\VerbBar}{|}
\newcommand{\VERB}{\Verb[commandchars=\\\{\}]}
\DefineVerbatimEnvironment{Highlighting}{Verbatim}{commandchars=\\\{\}}
% Add ',fontsize=\small' for more characters per line
\usepackage{framed}
\definecolor{shadecolor}{RGB}{248,248,248}
\newenvironment{Shaded}{\begin{snugshade}}{\end{snugshade}}
\newcommand{\AlertTok}[1]{\textcolor[rgb]{0.94,0.16,0.16}{#1}}
\newcommand{\AnnotationTok}[1]{\textcolor[rgb]{0.56,0.35,0.01}{\textbf{\textit{#1}}}}
\newcommand{\AttributeTok}[1]{\textcolor[rgb]{0.77,0.63,0.00}{#1}}
\newcommand{\BaseNTok}[1]{\textcolor[rgb]{0.00,0.00,0.81}{#1}}
\newcommand{\BuiltInTok}[1]{#1}
\newcommand{\CharTok}[1]{\textcolor[rgb]{0.31,0.60,0.02}{#1}}
\newcommand{\CommentTok}[1]{\textcolor[rgb]{0.56,0.35,0.01}{\textit{#1}}}
\newcommand{\CommentVarTok}[1]{\textcolor[rgb]{0.56,0.35,0.01}{\textbf{\textit{#1}}}}
\newcommand{\ConstantTok}[1]{\textcolor[rgb]{0.00,0.00,0.00}{#1}}
\newcommand{\ControlFlowTok}[1]{\textcolor[rgb]{0.13,0.29,0.53}{\textbf{#1}}}
\newcommand{\DataTypeTok}[1]{\textcolor[rgb]{0.13,0.29,0.53}{#1}}
\newcommand{\DecValTok}[1]{\textcolor[rgb]{0.00,0.00,0.81}{#1}}
\newcommand{\DocumentationTok}[1]{\textcolor[rgb]{0.56,0.35,0.01}{\textbf{\textit{#1}}}}
\newcommand{\ErrorTok}[1]{\textcolor[rgb]{0.64,0.00,0.00}{\textbf{#1}}}
\newcommand{\ExtensionTok}[1]{#1}
\newcommand{\FloatTok}[1]{\textcolor[rgb]{0.00,0.00,0.81}{#1}}
\newcommand{\FunctionTok}[1]{\textcolor[rgb]{0.00,0.00,0.00}{#1}}
\newcommand{\ImportTok}[1]{#1}
\newcommand{\InformationTok}[1]{\textcolor[rgb]{0.56,0.35,0.01}{\textbf{\textit{#1}}}}
\newcommand{\KeywordTok}[1]{\textcolor[rgb]{0.13,0.29,0.53}{\textbf{#1}}}
\newcommand{\NormalTok}[1]{#1}
\newcommand{\OperatorTok}[1]{\textcolor[rgb]{0.81,0.36,0.00}{\textbf{#1}}}
\newcommand{\OtherTok}[1]{\textcolor[rgb]{0.56,0.35,0.01}{#1}}
\newcommand{\PreprocessorTok}[1]{\textcolor[rgb]{0.56,0.35,0.01}{\textit{#1}}}
\newcommand{\RegionMarkerTok}[1]{#1}
\newcommand{\SpecialCharTok}[1]{\textcolor[rgb]{0.00,0.00,0.00}{#1}}
\newcommand{\SpecialStringTok}[1]{\textcolor[rgb]{0.31,0.60,0.02}{#1}}
\newcommand{\StringTok}[1]{\textcolor[rgb]{0.31,0.60,0.02}{#1}}
\newcommand{\VariableTok}[1]{\textcolor[rgb]{0.00,0.00,0.00}{#1}}
\newcommand{\VerbatimStringTok}[1]{\textcolor[rgb]{0.31,0.60,0.02}{#1}}
\newcommand{\WarningTok}[1]{\textcolor[rgb]{0.56,0.35,0.01}{\textbf{\textit{#1}}}}
\usepackage{longtable,booktabs}
\usepackage{calc} % for calculating minipage widths
% Correct order of tables after \paragraph or \subparagraph
\usepackage{etoolbox}
\makeatletter
\patchcmd\longtable{\par}{\if@noskipsec\mbox{}\fi\par}{}{}
\makeatother
% Allow footnotes in longtable head/foot
\IfFileExists{footnotehyper.sty}{\usepackage{footnotehyper}}{\usepackage{footnote}}
\makesavenoteenv{longtable}
\usepackage{graphicx}
\makeatletter
\def\maxwidth{\ifdim\Gin@nat@width>\linewidth\linewidth\else\Gin@nat@width\fi}
\def\maxheight{\ifdim\Gin@nat@height>\textheight\textheight\else\Gin@nat@height\fi}
\makeatother
% Scale images if necessary, so that they will not overflow the page
% margins by default, and it is still possible to overwrite the defaults
% using explicit options in \includegraphics[width, height, ...]{}
\setkeys{Gin}{width=\maxwidth,height=\maxheight,keepaspectratio}
% Set default figure placement to htbp
\makeatletter
\def\fps@figure{htbp}
\makeatother
\setlength{\emergencystretch}{3em} % prevent overfull lines
\providecommand{\tightlist}{%
  \setlength{\itemsep}{0pt}\setlength{\parskip}{0pt}}
\setcounter{secnumdepth}{-\maxdimen} % remove section numbering
\ifluatex
  \usepackage{selnolig}  % disable illegal ligatures
\fi

\title{Estudo de caso 03: Comparação de desempenho de duas configurações
de um algoritmo de otimização}
\author{Lucas Carneiro \and Vinicius Ferreira}
\date{Fevereiro 15, 2021}

\begin{document}
\maketitle

\hypertarget{sumuxe1rio}{%
\section{Sumário}\label{sumuxe1rio}}

A fim de solucionar problemas de otimização, heurísticas são utilizadas
a fim de encontrar uma solução viável em tempo hábil, dado que
algoritmos para encontrar a solução ótima exata levariam anos para obter
sucesso. Uma dessas heurísticas são algoritmos baseados em população,
que utilizam um conjunto de soluções-candidatas e, com uma série de
operações de variação e seleção nelas, exploram o espaço amostral do
problema em busca do ótimo. Neste ensaio, utiliza-se o método de
evolução diferencial.

Este trabalho explora como duas diferentes configurações
(\protect\hyperlink{tabel1}{tabela 1}) do método citado impactam no
resultado, observando qual destas obtém melhor desempenho. Para testar
as configurações, utiliza-se uma função de Rosenbrock, com o intervalo
de dimensões de interesse de \([2,150]\).

\begin{longtable}[]{@{}lll@{}}
\toprule
Configuração & Mutação & Recombinação\tabularnewline
\midrule
\endhead
1 & Aleatória (f = 4) & Aritmética\tabularnewline
2 & Melhores (f = 3) & Binária (cr = 0.7)\tabularnewline
\bottomrule
\end{longtable}

As próximas etapas deste estudo de caso é apresentado a seguir:

\begin{enumerate}
\def\labelenumi{\arabic{enumi}.}
\tightlist
\item
  Formulação das hipóteses de teste;
\item
  Cálculo dos tamanhos amostrais;
\item
  Coleta e tabulação dos dados;
\item
  Teste das hipóteses;
\item
  Estimação da magnitude da diferença entre os métodos;
\item
  Verificação das premissas dos testes;
\item
  Conclusões
\item
  Discussão sobre os processos do estudo
\end{enumerate}

\hypertarget{formulauxe7uxe3o-das-hipuxf3teses-de-teste}{%
\section{Formulação das hipóteses de
teste}\label{formulauxe7uxe3o-das-hipuxf3teses-de-teste}}

Como a proposta do estudo é de comparar como as modificações impactam
nos resultados, utiliza-se o valor médio de . Portanto, a primeira
hipótese é de comparar se os valores médios de cada uma das
configurações são iguais, ou seja:

\[
\begin{cases} H_0: \overline{\mu_1} = \overline{\mu_2}\\H_1: \overline{\mu_i} \neq \overline{\mu_2}\end{cases}
\]

Na hipótese definida acima, \(\mu_1\) e \(\mu_2\) correspondem aos
valores das médias das configurações 1 e 2, respectivamente.A hipótese
será testada utilizando o método de blocagem RCBD, assumindo que terá
uma replicação por bloco, independência dos blocos e independência em
aleatorização dentro dos blocos.

Para a hipóteses, utiliza-se um mínimo de importância prática de
\(0,5\), um índice de significância de \(\alpha = 0,05\) e potência
mínima desejada de \(\beta = 0.8\).

\hypertarget{descriuxe7uxe3o-de-dados-e-tratamento}{%
\section{Descrição de dados e
tratamento}\label{descriuxe7uxe3o-de-dados-e-tratamento}}

Nesta seção, discute-se sobre qual a quantidade de instâncias e
repetições utilizadas neste estudo. Para a avaliação das instâncias,
utilizou-se a biblioteca CAISEr para estimar a quantidade de instâncias
necessárias para cada configuração do algoritmo.

\begin{Shaded}
\begin{Highlighting}[]
\ControlFlowTok{if}\NormalTok{ (}\SpecialCharTok{!}\FunctionTok{require}\NormalTok{(}\StringTok{\textquotesingle{}multcomp\textquotesingle{}}\NormalTok{, }\AttributeTok{character.only =} \ConstantTok{TRUE}\NormalTok{)) \{}
      \FunctionTok{install.packages}\NormalTok{(}\StringTok{\textquotesingle{}multcomp\textquotesingle{}}\NormalTok{, }\AttributeTok{dependencies =} \ConstantTok{TRUE}\NormalTok{)}
      \FunctionTok{library}\NormalTok{(}\StringTok{\textquotesingle{}multcomp\textquotesingle{}}\NormalTok{, }\AttributeTok{character.only =} \ConstantTok{TRUE}\NormalTok{)}
\NormalTok{\}}
\end{Highlighting}
\end{Shaded}

\begin{verbatim}
## Loading required package: multcomp
\end{verbatim}

\begin{verbatim}
## Loading required package: mvtnorm
\end{verbatim}

\begin{verbatim}
## Loading required package: survival
\end{verbatim}

\begin{verbatim}
## Loading required package: TH.data
\end{verbatim}

\begin{verbatim}
## Loading required package: MASS
\end{verbatim}

\begin{verbatim}
## 
## Attaching package: 'TH.data'
\end{verbatim}

\begin{verbatim}
## The following object is masked from 'package:MASS':
## 
##     geyser
\end{verbatim}

\begin{Shaded}
\begin{Highlighting}[]
\ControlFlowTok{if}\NormalTok{ (}\SpecialCharTok{!}\FunctionTok{require}\NormalTok{(}\StringTok{\textquotesingle{}CAISEr\textquotesingle{}}\NormalTok{, }\AttributeTok{character.only =} \ConstantTok{TRUE}\NormalTok{)) \{}
      \FunctionTok{install.packages}\NormalTok{(}\StringTok{\textquotesingle{}CAISEr\textquotesingle{}}\NormalTok{, }\AttributeTok{dependencies =} \ConstantTok{TRUE}\NormalTok{)}
      \FunctionTok{library}\NormalTok{(}\StringTok{\textquotesingle{}CAISEr\textquotesingle{}}\NormalTok{, }\AttributeTok{character.only =} \ConstantTok{TRUE}\NormalTok{)}
\NormalTok{\}}
\end{Highlighting}
\end{Shaded}

\begin{verbatim}
## Loading required package: CAISEr
\end{verbatim}

\begin{verbatim}
## Warning: package 'CAISEr' was built under R version 4.0.4
\end{verbatim}

\begin{verbatim}
## 
## CAISEr version 1.0.16
## Not compatible with code developed for 0.X.Y versions
## If needed, please visit https://git.io/fjFwf for version 0.3.3
\end{verbatim}

\begin{Shaded}
\begin{Highlighting}[]
\CommentTok{\# Número de instancias}
\NormalTok{Ncalc }\OtherTok{\textless{}{-}} \FunctionTok{calc\_instances}\NormalTok{(}\DecValTok{1}\NormalTok{,}\AttributeTok{power =} \FloatTok{0.8}\NormalTok{, }\AttributeTok{d =} \FloatTok{0.5}\NormalTok{, }\AttributeTok{sig.level =} \FloatTok{0.05}\NormalTok{, }\AttributeTok{alternative =} \StringTok{"two.sided"}\NormalTok{, }\AttributeTok{test =} \StringTok{"t.test"}\NormalTok{)}
\NormalTok{Ncalc}
\end{Highlighting}
\end{Shaded}

\begin{verbatim}
## $ninstances
## [1] 34
## 
## $power
## [1] 0.8077767
## 
## $mean.power
## [1] 0.8077767
## 
## $median.power
## [1] 0.8077767
## 
## $d
## [1] 0.5
## 
## $sig.level
## [1] 0.05
## 
## $alternative
## [1] "two.sided"
## 
## $test
## [1] "t.test"
## 
## $power.target
## [1] "mean"
\end{verbatim}

Portanto, será utilizado 34 instâncias para cada configuração, o que
resulta em 68 instâncias no total.

Para gerar um valor médio confiável para cada instância, cada
configuração será repetida 30 vezes, pois esta é uma quantidade
considerável para se atingir tal média.

\hypertarget{coleta-e-tabulauxe7uxe3o-dos-dados}{%
\section{Coleta e tabulação dos
dados}\label{coleta-e-tabulauxe7uxe3o-dos-dados}}

Após o processo de coleta ser executado, os dados foram categorizados e
extraiu-se deles a média de 30 repetições de cada configuração, esta
apresentada abaixo.

\begin{Shaded}
\begin{Highlighting}[]
\NormalTok{fulldata }\OtherTok{\textless{}{-}} \FunctionTok{read.csv2}\NormalTok{(}\StringTok{"..}\SpecialCharTok{\textbackslash{}\textbackslash{}}\StringTok{data}\SpecialCharTok{\textbackslash{}\textbackslash{}}\StringTok{inst\_data\_complete.csv"}\NormalTok{, }\AttributeTok{sep=}\StringTok{\textquotesingle{};\textquotesingle{}}\NormalTok{, }\AttributeTok{header =} \ConstantTok{TRUE}\NormalTok{)}
\NormalTok{aggdata }\OtherTok{\textless{}{-}} \FunctionTok{with}\NormalTok{(fulldata, }\FunctionTok{aggregate}\NormalTok{(}\AttributeTok{x=}\NormalTok{resultado,}
                                    \AttributeTok{by=}\FunctionTok{list}\NormalTok{(config,instancia),}
                                    \AttributeTok{FUN=}\NormalTok{mean))}
\FunctionTok{names}\NormalTok{(aggdata) }\OtherTok{\textless{}{-}} \FunctionTok{c}\NormalTok{(}\StringTok{"Configuracao"}\NormalTok{,}\StringTok{"Instancia"}\NormalTok{,}\StringTok{"Resultado"}\NormalTok{)}
\ControlFlowTok{for}\NormalTok{(i }\ControlFlowTok{in} \DecValTok{1}\SpecialCharTok{:}\DecValTok{2}\NormalTok{) aggdata[,i] }\OtherTok{\textless{}{-}} \FunctionTok{as.factor}\NormalTok{(aggdata[,i])}
\FunctionTok{levels}\NormalTok{(aggdata}\SpecialCharTok{$}\NormalTok{Configuracao) }\OtherTok{\textless{}{-}} \FunctionTok{c}\NormalTok{(}\StringTok{"Config1"}\NormalTok{,}\StringTok{"Config2"}\NormalTok{)}
\FunctionTok{summary}\NormalTok{(aggdata)}
\end{Highlighting}
\end{Shaded}

\begin{verbatim}
##   Configuracao   Instancia    Resultado       
##  Config1:34    1      : 2   Min.   :       0  
##  Config2:34    2      : 2   1st Qu.:    6380  
##                3      : 2   Median :   98149  
##                4      : 2   Mean   : 2438282  
##                5      : 2   3rd Qu.: 4575638  
##                6      : 2   Max.   :11623286  
##                (Other):56
\end{verbatim}

\begin{Shaded}
\begin{Highlighting}[]
\FunctionTok{library}\NormalTok{(ggplot2)}
\NormalTok{p }\OtherTok{\textless{}{-}} \FunctionTok{ggplot}\NormalTok{(aggdata, }\FunctionTok{aes}\NormalTok{(}\AttributeTok{x =}\NormalTok{ Instancia, }
                         \AttributeTok{y =}\NormalTok{ Resultado, }
                         \AttributeTok{group =}\NormalTok{ Configuracao, }
                         \AttributeTok{colour =}\NormalTok{ Configuracao))}
\NormalTok{p }\SpecialCharTok{+} \FunctionTok{geom\_line}\NormalTok{(}\AttributeTok{linetype =} \DecValTok{2}\NormalTok{) }\SpecialCharTok{+} \FunctionTok{geom\_point}\NormalTok{(}\AttributeTok{size=}\DecValTok{3}\NormalTok{)}
\end{Highlighting}
\end{Shaded}

\includegraphics{estudo_03_files/figure-latex/unnamed-chunk-10-1.pdf}

Observa-se que as observações com relação à configuração 1 demonstram um
aumento significativo de seus valores, enquanto a média dos resultados
com a segunda configuração permanecem baixos, ou seja, ja há indicações
que não podem ser consideradas os mesmos valores de médias.

\hypertarget{teste-das-hipuxf3teses}{%
\section{Teste das hipóteses}\label{teste-das-hipuxf3teses}}

\begin{Shaded}
\begin{Highlighting}[]
\CommentTok{\# Análise Anova }
\NormalTok{mdl1 }\OtherTok{\textless{}{-}} \FunctionTok{aov}\NormalTok{(Resultado}\SpecialCharTok{\textasciitilde{}}\NormalTok{Configuracao}\SpecialCharTok{+}\NormalTok{Instancia, }\AttributeTok{data =}\NormalTok{ aggdata)}
\NormalTok{aggdata}\SpecialCharTok{$}\NormalTok{Resultado[}\DecValTok{1}\NormalTok{] }\OtherTok{\textless{}{-}} \FloatTok{0.0001} \CommentTok{\# Log de 0 dá {-}inf}
\NormalTok{mdl2 }\OtherTok{\textless{}{-}} \FunctionTok{aov}\NormalTok{(}\FunctionTok{log}\NormalTok{(Resultado)}\SpecialCharTok{\textasciitilde{}}\NormalTok{Configuracao}\SpecialCharTok{+}\NormalTok{Instancia, }\AttributeTok{data =}\NormalTok{ aggdata)}
\FunctionTok{summary}\NormalTok{(mdl1)}
\end{Highlighting}
\end{Shaded}

\begin{verbatim}
##              Df    Sum Sq   Mean Sq F value   Pr(>F)    
## Configuracao  1 3.935e+14 3.935e+14  55.836 1.37e-08 ***
## Instancia    33 2.423e+14 7.341e+12   1.042    0.454    
## Residuals    33 2.326e+14 7.047e+12                     
## ---
## Signif. codes:  0 '***' 0.001 '**' 0.01 '*' 0.05 '.' 0.1 ' ' 1
\end{verbatim}

\begin{Shaded}
\begin{Highlighting}[]
\FunctionTok{summary}\NormalTok{(mdl2)}
\end{Highlighting}
\end{Shaded}

\begin{verbatim}
##              Df Sum Sq Mean Sq F value   Pr(>F)    
## Configuracao  1  689.4   689.4  503.21  < 2e-16 ***
## Instancia    33  943.3    28.6   20.87 2.34e-14 ***
## Residuals    33   45.2     1.4                     
## ---
## Signif. codes:  0 '***' 0.001 '**' 0.01 '*' 0.05 '.' 0.1 ' ' 1
\end{verbatim}

\begin{Shaded}
\begin{Highlighting}[]
\FunctionTok{par}\NormalTok{(}\AttributeTok{mfrow =} \FunctionTok{c}\NormalTok{(}\DecValTok{2}\NormalTok{, }\DecValTok{2}\NormalTok{))}
\FunctionTok{plot}\NormalTok{(mdl1, }\AttributeTok{pch =} \DecValTok{20}\NormalTok{, }\AttributeTok{las =} \DecValTok{1}\NormalTok{)}
\end{Highlighting}
\end{Shaded}

\includegraphics{estudo_03_files/figure-latex/unnamed-chunk-11-1.pdf}

\begin{Shaded}
\begin{Highlighting}[]
\FunctionTok{par}\NormalTok{(}\AttributeTok{mfrow =} \FunctionTok{c}\NormalTok{(}\DecValTok{2}\NormalTok{, }\DecValTok{2}\NormalTok{))}
\FunctionTok{plot}\NormalTok{(mdl2, }\AttributeTok{pch =} \DecValTok{20}\NormalTok{, }\AttributeTok{las =} \DecValTok{1}\NormalTok{)}
\end{Highlighting}
\end{Shaded}

\includegraphics{estudo_03_files/figure-latex/unnamed-chunk-11-2.pdf}

\hypertarget{estimauxe7uxe3o-da-magnitude-da-diferenuxe7a-entre-os-muxe9todos}{%
\section{Estimação da magnitude da diferença entre os
métodos}\label{estimauxe7uxe3o-da-magnitude-da-diferenuxe7a-entre-os-muxe9todos}}

\begin{Shaded}
\begin{Highlighting}[]
\CommentTok{\# Comparação Múltipla (caso dê diferença)}
\NormalTok{dtest1 }\OtherTok{\textless{}{-}} \FunctionTok{glht}\NormalTok{(mdl1, }\AttributeTok{linfct =} \FunctionTok{mcp}\NormalTok{(}\AttributeTok{Configuracao =} \StringTok{"Dunnet"}\NormalTok{))}
\NormalTok{dtestIC1 }\OtherTok{\textless{}{-}} \FunctionTok{confint}\NormalTok{(dtest1,}\AttributeTok{level =} \FloatTok{0.95}\NormalTok{)}
\FunctionTok{par}\NormalTok{(}\AttributeTok{mar =} \FunctionTok{c}\NormalTok{(}\DecValTok{5}\NormalTok{, }\DecValTok{8}\NormalTok{, }\DecValTok{4}\NormalTok{, }\DecValTok{2}\NormalTok{), }\AttributeTok{las =} \DecValTok{1}\NormalTok{)}
\FunctionTok{plot}\NormalTok{(dtestIC1,}\AttributeTok{xlim =} \FunctionTok{c}\NormalTok{(}\SpecialCharTok{{-}}\DecValTok{10}\SpecialCharTok{\^{}}\DecValTok{7}\NormalTok{,}\DecValTok{10}\SpecialCharTok{\^{}}\DecValTok{7}\NormalTok{))}
\end{Highlighting}
\end{Shaded}

\includegraphics{estudo_03_files/figure-latex/unnamed-chunk-12-1.pdf}

\begin{Shaded}
\begin{Highlighting}[]
\NormalTok{dtest2 }\OtherTok{\textless{}{-}} \FunctionTok{glht}\NormalTok{(mdl2, }\AttributeTok{linfct =} \FunctionTok{mcp}\NormalTok{(}\AttributeTok{Configuracao =} \StringTok{"Dunnet"}\NormalTok{))}
\NormalTok{dtestIC2 }\OtherTok{\textless{}{-}} \FunctionTok{confint}\NormalTok{(dtest2,}\AttributeTok{level =} \FloatTok{0.95}\NormalTok{)}
\FunctionTok{par}\NormalTok{(}\AttributeTok{mar =} \FunctionTok{c}\NormalTok{(}\DecValTok{5}\NormalTok{, }\DecValTok{8}\NormalTok{, }\DecValTok{4}\NormalTok{, }\DecValTok{2}\NormalTok{), }\AttributeTok{las =} \DecValTok{1}\NormalTok{)}
\FunctionTok{plot}\NormalTok{(dtestIC2,}\AttributeTok{xlim =} \FunctionTok{c}\NormalTok{(}\SpecialCharTok{{-}}\DecValTok{10}\NormalTok{,}\DecValTok{10}\NormalTok{))}
\end{Highlighting}
\end{Shaded}

\includegraphics{estudo_03_files/figure-latex/unnamed-chunk-12-2.pdf}

\hypertarget{verificauxe7uxe3o-das-premissas-dos-testes}{%
\section{Verificação das premissas dos
testes}\label{verificauxe7uxe3o-das-premissas-dos-testes}}

Como a proposta do estudo é de comparar como as modificações impactam
nos resultados, utiliza-se o valor médio de . Portanto, a primeira
hipótese é de comparar se os valores médios de cada uma das
configurações são iguais, ou seja:

\[
\begin{cases} H_0: \overline{\mu_1} = \overline{\mu_2}\\H_1: \overline{\mu_i} \neq \overline{\mu_2}\end{cases}
\]

Na hipótese definida acima, \(\mu_1\) e \(\mu_2\) correspondem aos
valores das médias das configurações 1 e 2, respectivamente.A hipótese
será testada utilizando o método de blocagem RCBD, assumindo que terá
uma replicação por bloco, independência dos blocos e independência em
aleatorização dentro dos blocos.

Para a hipóteses, utiliza-se um mínimo de importância prática de
\(0,5\), um índice de significância de \(\alpha = 0,05\) e potência
mínima desejada de \(\beta = 0.8\).

\hypertarget{conclusuxe3o-e-recomendauxe7uxe3o}{%
\section{Conclusão e
recomendação}\label{conclusuxe3o-e-recomendauxe7uxe3o}}

Como a proposta do estudo é de comparar como as modificações impactam
nos resultados, utiliza-se o valor médio de . Portanto, a primeira
hipótese é de comparar se os valores médios de cada uma das
configurações são iguais, ou seja:

\[
\begin{cases} H_0: \overline{\mu_1} = \overline{\mu_2}\\H_1: \overline{\mu_i} \neq \overline{\mu_2}\end{cases}
\]

Na hipótese definida acima, \(\mu_1\) e \(\mu_2\) correspondem aos
valores das médias das configurações 1 e 2, respectivamente.A hipótese
será testada utilizando o método de blocagem RCBD, assumindo que terá
uma replicação por bloco, independência dos blocos e independência em
aleatorização dentro dos blocos.

Para a hipóteses, utiliza-se um mínimo de importância prática de
\(0,5\), um índice de significância de \(\alpha = 0,05\) e potência
mínima desejada de \(\beta = 0.8\).

\end{document}
