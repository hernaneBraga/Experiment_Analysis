% Options for packages loaded elsewhere
\PassOptionsToPackage{unicode}{hyperref}
\PassOptionsToPackage{hyphens}{url}
%
\documentclass[
]{article}
\usepackage{lmodern}
\usepackage{amsmath}
\usepackage{ifxetex,ifluatex}
\ifnum 0\ifxetex 1\fi\ifluatex 1\fi=0 % if pdftex
  \usepackage[T1]{fontenc}
  \usepackage[utf8]{inputenc}
  \usepackage{textcomp} % provide euro and other symbols
  \usepackage{amssymb}
\else % if luatex or xetex
  \usepackage{unicode-math}
  \defaultfontfeatures{Scale=MatchLowercase}
  \defaultfontfeatures[\rmfamily]{Ligatures=TeX,Scale=1}
\fi
% Use upquote if available, for straight quotes in verbatim environments
\IfFileExists{upquote.sty}{\usepackage{upquote}}{}
\IfFileExists{microtype.sty}{% use microtype if available
  \usepackage[]{microtype}
  \UseMicrotypeSet[protrusion]{basicmath} % disable protrusion for tt fonts
}{}
\makeatletter
\@ifundefined{KOMAClassName}{% if non-KOMA class
  \IfFileExists{parskip.sty}{%
    \usepackage{parskip}
  }{% else
    \setlength{\parindent}{0pt}
    \setlength{\parskip}{6pt plus 2pt minus 1pt}}
}{% if KOMA class
  \KOMAoptions{parskip=half}}
\makeatother
\usepackage{xcolor}
\IfFileExists{xurl.sty}{\usepackage{xurl}}{} % add URL line breaks if available
\IfFileExists{bookmark.sty}{\usepackage{bookmark}}{\usepackage{hyperref}}
\hypersetup{
  pdftitle={Estudo de caso 03: Comparação de desempenho de duas configurações de um algoritmo de otimização},
  pdfauthor={Lucas Carneiro - Análises; Vinicius Ferreira - Coleta de dados},
  hidelinks,
  pdfcreator={LaTeX via pandoc}}
\urlstyle{same} % disable monospaced font for URLs
\usepackage[margin=1in]{geometry}
\usepackage{color}
\usepackage{fancyvrb}
\newcommand{\VerbBar}{|}
\newcommand{\VERB}{\Verb[commandchars=\\\{\}]}
\DefineVerbatimEnvironment{Highlighting}{Verbatim}{commandchars=\\\{\}}
% Add ',fontsize=\small' for more characters per line
\usepackage{framed}
\definecolor{shadecolor}{RGB}{248,248,248}
\newenvironment{Shaded}{\begin{snugshade}}{\end{snugshade}}
\newcommand{\AlertTok}[1]{\textcolor[rgb]{0.94,0.16,0.16}{#1}}
\newcommand{\AnnotationTok}[1]{\textcolor[rgb]{0.56,0.35,0.01}{\textbf{\textit{#1}}}}
\newcommand{\AttributeTok}[1]{\textcolor[rgb]{0.77,0.63,0.00}{#1}}
\newcommand{\BaseNTok}[1]{\textcolor[rgb]{0.00,0.00,0.81}{#1}}
\newcommand{\BuiltInTok}[1]{#1}
\newcommand{\CharTok}[1]{\textcolor[rgb]{0.31,0.60,0.02}{#1}}
\newcommand{\CommentTok}[1]{\textcolor[rgb]{0.56,0.35,0.01}{\textit{#1}}}
\newcommand{\CommentVarTok}[1]{\textcolor[rgb]{0.56,0.35,0.01}{\textbf{\textit{#1}}}}
\newcommand{\ConstantTok}[1]{\textcolor[rgb]{0.00,0.00,0.00}{#1}}
\newcommand{\ControlFlowTok}[1]{\textcolor[rgb]{0.13,0.29,0.53}{\textbf{#1}}}
\newcommand{\DataTypeTok}[1]{\textcolor[rgb]{0.13,0.29,0.53}{#1}}
\newcommand{\DecValTok}[1]{\textcolor[rgb]{0.00,0.00,0.81}{#1}}
\newcommand{\DocumentationTok}[1]{\textcolor[rgb]{0.56,0.35,0.01}{\textbf{\textit{#1}}}}
\newcommand{\ErrorTok}[1]{\textcolor[rgb]{0.64,0.00,0.00}{\textbf{#1}}}
\newcommand{\ExtensionTok}[1]{#1}
\newcommand{\FloatTok}[1]{\textcolor[rgb]{0.00,0.00,0.81}{#1}}
\newcommand{\FunctionTok}[1]{\textcolor[rgb]{0.00,0.00,0.00}{#1}}
\newcommand{\ImportTok}[1]{#1}
\newcommand{\InformationTok}[1]{\textcolor[rgb]{0.56,0.35,0.01}{\textbf{\textit{#1}}}}
\newcommand{\KeywordTok}[1]{\textcolor[rgb]{0.13,0.29,0.53}{\textbf{#1}}}
\newcommand{\NormalTok}[1]{#1}
\newcommand{\OperatorTok}[1]{\textcolor[rgb]{0.81,0.36,0.00}{\textbf{#1}}}
\newcommand{\OtherTok}[1]{\textcolor[rgb]{0.56,0.35,0.01}{#1}}
\newcommand{\PreprocessorTok}[1]{\textcolor[rgb]{0.56,0.35,0.01}{\textit{#1}}}
\newcommand{\RegionMarkerTok}[1]{#1}
\newcommand{\SpecialCharTok}[1]{\textcolor[rgb]{0.00,0.00,0.00}{#1}}
\newcommand{\SpecialStringTok}[1]{\textcolor[rgb]{0.31,0.60,0.02}{#1}}
\newcommand{\StringTok}[1]{\textcolor[rgb]{0.31,0.60,0.02}{#1}}
\newcommand{\VariableTok}[1]{\textcolor[rgb]{0.00,0.00,0.00}{#1}}
\newcommand{\VerbatimStringTok}[1]{\textcolor[rgb]{0.31,0.60,0.02}{#1}}
\newcommand{\WarningTok}[1]{\textcolor[rgb]{0.56,0.35,0.01}{\textbf{\textit{#1}}}}
\usepackage{longtable,booktabs}
\usepackage{calc} % for calculating minipage widths
% Correct order of tables after \paragraph or \subparagraph
\usepackage{etoolbox}
\makeatletter
\patchcmd\longtable{\par}{\if@noskipsec\mbox{}\fi\par}{}{}
\makeatother
% Allow footnotes in longtable head/foot
\IfFileExists{footnotehyper.sty}{\usepackage{footnotehyper}}{\usepackage{footnote}}
\makesavenoteenv{longtable}
\usepackage{graphicx}
\makeatletter
\def\maxwidth{\ifdim\Gin@nat@width>\linewidth\linewidth\else\Gin@nat@width\fi}
\def\maxheight{\ifdim\Gin@nat@height>\textheight\textheight\else\Gin@nat@height\fi}
\makeatother
% Scale images if necessary, so that they will not overflow the page
% margins by default, and it is still possible to overwrite the defaults
% using explicit options in \includegraphics[width, height, ...]{}
\setkeys{Gin}{width=\maxwidth,height=\maxheight,keepaspectratio}
% Set default figure placement to htbp
\makeatletter
\def\fps@figure{htbp}
\makeatother
\setlength{\emergencystretch}{3em} % prevent overfull lines
\providecommand{\tightlist}{%
  \setlength{\itemsep}{0pt}\setlength{\parskip}{0pt}}
\setcounter{secnumdepth}{-\maxdimen} % remove section numbering
\ifluatex
  \usepackage{selnolig}  % disable illegal ligatures
\fi

\title{Estudo de caso 03: Comparação de desempenho de duas configurações
de um algoritmo de otimização}
\author{Lucas Carneiro - Análises \and Vinicius Ferreira - Coleta de
dados}
\date{Março 15, 2021}

\begin{document}
\maketitle

\hypertarget{sumuxe1rio}{%
\section{Sumário}\label{sumuxe1rio}}

A fim de solucionar problemas de otimização, heurísticas são utilizadas
a fim de encontrar uma solução viável em tempo hábil, dado que
algoritmos para encontrar a solução ótima exata levariam anos para obter
sucesso. Uma dessas heurísticas são algoritmos baseados em população,
que utilizam um conjunto de soluções-candidatas e, com uma série de
operações de variação e seleção nelas, exploram o espaço amostral do
problema em busca do ótimo. Neste ensaio, utiliza-se o método de
evolução diferencial.

Este trabalho explora como duas diferentes configurações
(\protect\hyperlink{table1}{Table 1}) do método citado impactam no
resultado, observando qual destas obtém melhor desempenho. Para testar
as configurações, utiliza-se uma função de Rosenbrock, com o intervalo
de dimensões de interesse de \([2,150]\).

\begin{longtable}[]{@{}lll@{}}
\caption{Características das configurações}\tabularnewline
\toprule
Configuração & Mutação & Recombinação\tabularnewline
\midrule
\endfirsthead
\toprule
Configuração & Mutação & Recombinação\tabularnewline
\midrule
\endhead
1 & Aleatória (f = 4) & Aritmética\tabularnewline
2 & Melhores (f = 3) & Binária (cr = 0.7)\tabularnewline
\bottomrule
\end{longtable}

Este estudo é organizado como indicado a seguir:

\begin{enumerate}
\def\labelenumi{\arabic{enumi}.}
\tightlist
\item
  Formulação das hipóteses de teste;
\item
  Coleta e Tratamento de Dados;
\item
  Resultados e Discussão -- Teste das hipóteses;
\item
  Conclusões
\item
  Discussão sobre os processos do estudo
\end{enumerate}

\hypertarget{formulauxe7uxe3o-das-hipuxf3teses-de-teste}{%
\section{Formulação das hipóteses de
teste}\label{formulauxe7uxe3o-das-hipuxf3teses-de-teste}}

Como a proposta do estudo é comparar como as modificações impactam nos
resultados, utiliza-se como métrica de desempenho o valor médio da
função custo. Portanto, a primeira hipótese é de comparar se os valores
médios obtidos por cada uma das configurações são iguais, ou seja:

\[
\begin{cases} H_0: \overline{\mu_1} = \overline{\mu_2}\\H_1: \overline{\mu_i} \neq \overline{\mu_2}\end{cases}
\]

Na hipótese definida acima, \(\mu_1\) e \(\mu_2\) correspondem aos
valores das médias das configurações 1 e 2, respectivamente. A hipótese
será testada utilizando o método de blocagem RCBD, assumindo que terá
uma replicação por bloco, independência dos blocos e independência em
aleatorização dentro dos blocos.

Para a hipóteses, utiliza-se um mínimo de importância prática
padronizada de \(0,5\), um índice de significância de \(\alpha = 0,05\)
e potência mínima desejada de \(\beta = 0.8\).

\hypertarget{coleta-e-tratamento-de-dados}{%
\section{Coleta e Tratamento de
Dados}\label{coleta-e-tratamento-de-dados}}

Nesta seção, discute-se sobre qual a quantidade de instâncias e
repetições utilizadas neste estudo. Para a avaliação das instâncias,
utilizou-se a biblioteca CAISEr para estimar a quantidade de instâncias
necessárias para cada configuração do algoritmo.

\begin{Shaded}
\begin{Highlighting}[]
\ControlFlowTok{if}\NormalTok{ (}\SpecialCharTok{!}\FunctionTok{require}\NormalTok{(}\StringTok{\textquotesingle{}multcomp\textquotesingle{}}\NormalTok{, }\AttributeTok{character.only =} \ConstantTok{TRUE}\NormalTok{)) \{}
      \FunctionTok{install.packages}\NormalTok{(}\StringTok{\textquotesingle{}multcomp\textquotesingle{}}\NormalTok{, }\AttributeTok{dependencies =} \ConstantTok{TRUE}\NormalTok{)}
      \FunctionTok{library}\NormalTok{(}\StringTok{\textquotesingle{}multcomp\textquotesingle{}}\NormalTok{, }\AttributeTok{character.only =} \ConstantTok{TRUE}\NormalTok{)}
\NormalTok{\}}

\ControlFlowTok{if}\NormalTok{ (}\SpecialCharTok{!}\FunctionTok{require}\NormalTok{(}\StringTok{\textquotesingle{}CAISEr\textquotesingle{}}\NormalTok{, }\AttributeTok{character.only =} \ConstantTok{TRUE}\NormalTok{)) \{}
      \FunctionTok{install.packages}\NormalTok{(}\StringTok{\textquotesingle{}CAISEr\textquotesingle{}}\NormalTok{, }\AttributeTok{dependencies =} \ConstantTok{TRUE}\NormalTok{)}
      \FunctionTok{library}\NormalTok{(}\StringTok{\textquotesingle{}CAISEr\textquotesingle{}}\NormalTok{, }\AttributeTok{character.only =} \ConstantTok{TRUE}\NormalTok{)}
\NormalTok{\}}
\end{Highlighting}
\end{Shaded}

\begin{verbatim}
## Warning: package 'CAISEr' was built under R version 4.0.4
\end{verbatim}

\begin{Shaded}
\begin{Highlighting}[]
\CommentTok{\# Número de instancias}
\NormalTok{Ncalc }\OtherTok{\textless{}{-}} \FunctionTok{calc\_instances}\NormalTok{(}\DecValTok{1}\NormalTok{,}\AttributeTok{power =} \FloatTok{0.8}\NormalTok{, }\AttributeTok{d =} \FloatTok{0.5}\NormalTok{, }\AttributeTok{sig.level =} \FloatTok{0.05}\NormalTok{, }\AttributeTok{alternative =} \StringTok{"two.sided"}\NormalTok{, }\AttributeTok{test =} \StringTok{"t.test"}\NormalTok{)}
\NormalTok{Ncalc}
\end{Highlighting}
\end{Shaded}

\begin{verbatim}
## $ninstances
## [1] 34
## 
## $power
## [1] 0.8077767
## 
## $mean.power
## [1] 0.8077767
## 
## $median.power
## [1] 0.8077767
## 
## $d
## [1] 0.5
## 
## $sig.level
## [1] 0.05
## 
## $alternative
## [1] "two.sided"
## 
## $test
## [1] "t.test"
## 
## $power.target
## [1] "mean"
\end{verbatim}

Portanto, serão utilizadas 34 instâncias para cada configuração. Para
gerar um valor médio confiável em cada instância, serão realizadas 30
execuções para cada uma, este que é um valor ``padrão'' na literatura.

As instâncias foram definidas ao amostrar o intervalo de dimensões
\(([2,150])\) para o problema de Rosenbrock uniformemente. Assim, o
conjunto de dimensões escolhidas foi:

\(\{2,6,11,15,20,24,29,33,38,42,47,51,56,60,65,69,74,78,83,87,\\ 92,96,101,105,110,114,119,123,128,132,137,141,146,150\}\).

Após o processo de coleta ser executado, os dados foram categorizados e
extraiu-se deles a média das 30 repetições de cada configuração, esta
apresentada abaixo.

\begin{Shaded}
\begin{Highlighting}[]
\NormalTok{fulldata }\OtherTok{\textless{}{-}} \FunctionTok{read.csv2}\NormalTok{(}\StringTok{"..}\SpecialCharTok{\textbackslash{}\textbackslash{}}\StringTok{data}\SpecialCharTok{\textbackslash{}\textbackslash{}}\StringTok{inst\_data\_complete.csv"}\NormalTok{, }\AttributeTok{sep=}\StringTok{\textquotesingle{};\textquotesingle{}}\NormalTok{, }\AttributeTok{header =} \ConstantTok{TRUE}\NormalTok{)}
\NormalTok{aggdata }\OtherTok{\textless{}{-}} \FunctionTok{with}\NormalTok{(fulldata, }\FunctionTok{aggregate}\NormalTok{(}\AttributeTok{x=}\NormalTok{resultado,}
                                    \AttributeTok{by=}\FunctionTok{list}\NormalTok{(config,instancia),}
                                    \AttributeTok{FUN=}\NormalTok{mean))}
\FunctionTok{names}\NormalTok{(aggdata) }\OtherTok{\textless{}{-}} \FunctionTok{c}\NormalTok{(}\StringTok{"Configuracao"}\NormalTok{,}\StringTok{"Instancia"}\NormalTok{,}\StringTok{"Resultado"}\NormalTok{)}
\ControlFlowTok{for}\NormalTok{(i }\ControlFlowTok{in} \DecValTok{1}\SpecialCharTok{:}\DecValTok{2}\NormalTok{) aggdata[,i] }\OtherTok{\textless{}{-}} \FunctionTok{as.factor}\NormalTok{(aggdata[,i])}
\FunctionTok{levels}\NormalTok{(aggdata}\SpecialCharTok{$}\NormalTok{Configuracao) }\OtherTok{\textless{}{-}} \FunctionTok{c}\NormalTok{(}\StringTok{"Config1"}\NormalTok{,}\StringTok{"Config2"}\NormalTok{)}
\FunctionTok{summary}\NormalTok{(aggdata)}
\end{Highlighting}
\end{Shaded}

\begin{verbatim}
##   Configuracao   Instancia    Resultado       
##  Config1:34    1      : 2   Min.   :       0  
##  Config2:34    2      : 2   1st Qu.:    6380  
##                3      : 2   Median :   98149  
##                4      : 2   Mean   : 2438282  
##                5      : 2   3rd Qu.: 4575638  
##                6      : 2   Max.   :11623286  
##                (Other):56
\end{verbatim}

Depois de realizar a formatação dos dados, um gráfico da Média de
Desempenho por Instância para cada Configuração é apresentado.

\begin{Shaded}
\begin{Highlighting}[]
\FunctionTok{library}\NormalTok{(ggplot2)}
\NormalTok{p }\OtherTok{\textless{}{-}} \FunctionTok{ggplot}\NormalTok{(aggdata, }\FunctionTok{aes}\NormalTok{(}\AttributeTok{x =}\NormalTok{ Instancia, }
                         \AttributeTok{y =}\NormalTok{ Resultado, }
                         \AttributeTok{group =}\NormalTok{ Configuracao, }
                         \AttributeTok{colour =}\NormalTok{ Configuracao))}
\NormalTok{p }\SpecialCharTok{+} \FunctionTok{geom\_line}\NormalTok{(}\AttributeTok{linetype =} \DecValTok{2}\NormalTok{) }\SpecialCharTok{+} \FunctionTok{geom\_point}\NormalTok{(}\AttributeTok{size=}\DecValTok{3}\NormalTok{)}
\end{Highlighting}
\end{Shaded}

\includegraphics{estudo_03_files/figure-latex/data1_plot-1.pdf}

Infere-se que as observações da Configuração 2 demonstram um aumento
significativo de seus valores, enquanto a média dos resultados da
Configuração 1 permanecem baixos. Dessa forma, há indicações que as
médias não podem ser consideradas iguais.

Por conta da ordem de grandeza dos valores aferidos, além de fatores que
serão detalhados posteriormente, escolheu-se por realizar uma
transformação logarítmica nos dados. Assim, um novo gráfico de
Desempenho por Instancia é exibido em sequência.

\begin{Shaded}
\begin{Highlighting}[]
\NormalTok{aggdata2}\OtherTok{\textless{}{-}}\NormalTok{ aggdata}
\NormalTok{aggdata2}\SpecialCharTok{$}\NormalTok{Resultado[}\DecValTok{1}\NormalTok{] }\OtherTok{\textless{}{-}} \FloatTok{0.0001} \CommentTok{\# Log de 0 dá {-}inf}
\NormalTok{aggdata2}\SpecialCharTok{$}\NormalTok{Resultado }\OtherTok{\textless{}{-}} \FunctionTok{log}\NormalTok{(aggdata2}\SpecialCharTok{$}\NormalTok{Resultado)}
\NormalTok{p }\OtherTok{\textless{}{-}} \FunctionTok{ggplot}\NormalTok{(aggdata2, }\FunctionTok{aes}\NormalTok{(}\AttributeTok{x =}\NormalTok{ Instancia, }
                         \AttributeTok{y =}\NormalTok{ Resultado, }
                         \AttributeTok{group =}\NormalTok{ Configuracao, }
                         \AttributeTok{colour =}\NormalTok{ Configuracao))}
\NormalTok{p }\SpecialCharTok{+} \FunctionTok{geom\_line}\NormalTok{(}\AttributeTok{linetype =} \DecValTok{2}\NormalTok{) }\SpecialCharTok{+} \FunctionTok{geom\_point}\NormalTok{(}\AttributeTok{size=}\DecValTok{3}\NormalTok{)}
\end{Highlighting}
\end{Shaded}

\includegraphics{estudo_03_files/figure-latex/unnamed-chunk-6-1.pdf}

No gráfico acima é possível observar com maior detalhamento que a
Configuração 1 apresenta desempenho superior à Configuração 2 em cada um
dos cenários (instâncias) avaliados. Contudo, para uma análise mais
rigorosa, realiza-se o teste estatístico.

\hypertarget{resultados-e-discussuxe3o-teste-das-hipuxf3teses}{%
\section{Resultados e Discussão -- Teste das
Hipóteses}\label{resultados-e-discussuxe3o-teste-das-hipuxf3teses}}

O teste ANOVA para os dados obtidos e os gráficos dos resíduos são
apresentados a seguir:

\begin{Shaded}
\begin{Highlighting}[]
\CommentTok{\# Análise Anova }
\NormalTok{mdl1 }\OtherTok{\textless{}{-}} \FunctionTok{aov}\NormalTok{(Resultado}\SpecialCharTok{\textasciitilde{}}\NormalTok{Configuracao}\SpecialCharTok{+}\NormalTok{Instancia, }\AttributeTok{data =}\NormalTok{ aggdata)}
\FunctionTok{summary}\NormalTok{(mdl1)}
\end{Highlighting}
\end{Shaded}

\begin{verbatim}
##              Df    Sum Sq   Mean Sq F value   Pr(>F)    
## Configuracao  1 3.935e+14 3.935e+14  55.836 1.37e-08 ***
## Instancia    33 2.423e+14 7.341e+12   1.042    0.454    
## Residuals    33 2.326e+14 7.047e+12                     
## ---
## Signif. codes:  0 '***' 0.001 '**' 0.01 '*' 0.05 '.' 0.1 ' ' 1
\end{verbatim}

\begin{Shaded}
\begin{Highlighting}[]
\FunctionTok{summary.lm}\NormalTok{(mdl1)}\SpecialCharTok{$}\NormalTok{r.squared}
\end{Highlighting}
\end{Shaded}

\begin{verbatim}
## [1] 0.7321722
\end{verbatim}

\begin{Shaded}
\begin{Highlighting}[]
\FunctionTok{par}\NormalTok{(}\AttributeTok{mfrow =} \FunctionTok{c}\NormalTok{(}\DecValTok{2}\NormalTok{, }\DecValTok{2}\NormalTok{))}
\FunctionTok{plot}\NormalTok{(mdl1, }\AttributeTok{pch =} \DecValTok{20}\NormalTok{, }\AttributeTok{las =} \DecValTok{1}\NormalTok{)}
\end{Highlighting}
\end{Shaded}

\includegraphics{estudo_03_files/figure-latex/anova1-1.pdf}

Verifica-se que o teste rejeita a hipótese nula e, portanto, existe
diferença entre os desempenhos. Contudo, é possível observar que o
modelo gerado não apresenta um bom valor de coeficiente de determinação
\((R^2)\). A comparação de Dunnet entre as configurações, com intervalo
de confiância de \(95\%\), é apresentada a seguir.

\begin{Shaded}
\begin{Highlighting}[]
\CommentTok{\# Comparação Múltipla 1}
\NormalTok{dtest1 }\OtherTok{\textless{}{-}} \FunctionTok{glht}\NormalTok{(mdl1, }\AttributeTok{linfct =} \FunctionTok{mcp}\NormalTok{(}\AttributeTok{Configuracao =} \StringTok{"Dunnet"}\NormalTok{))}
\NormalTok{dtestIC1 }\OtherTok{\textless{}{-}} \FunctionTok{confint}\NormalTok{(dtest1,}\AttributeTok{level =} \FloatTok{0.95}\NormalTok{)}
\FunctionTok{par}\NormalTok{(}\AttributeTok{mar =} \FunctionTok{c}\NormalTok{(}\DecValTok{5}\NormalTok{, }\DecValTok{8}\NormalTok{, }\DecValTok{4}\NormalTok{, }\DecValTok{2}\NormalTok{), }\AttributeTok{las =} \DecValTok{1}\NormalTok{)}
\FunctionTok{plot}\NormalTok{(dtestIC1,}\AttributeTok{xlim =} \FunctionTok{c}\NormalTok{(}\SpecialCharTok{{-}}\DecValTok{10}\SpecialCharTok{\^{}}\DecValTok{7}\NormalTok{,}\DecValTok{10}\SpecialCharTok{\^{}}\DecValTok{7}\NormalTok{),}\AttributeTok{xlab =} \StringTok{"Diferença das Médias"}\NormalTok{)}
\end{Highlighting}
\end{Shaded}

\includegraphics{estudo_03_files/figure-latex/comp1-1.pdf}

Aqui, infere-se que há diferença entre o desempenho dos algoritmos. Além
disso, a Configuração 2 apresenta desempenho médio inferior (maior valor
de custo) à Configuração 1 para o problema em estudo. A magnitude das
diferenças observadas é da ordem de \(10^6\) em termos de função custo.

Para contornar o baixo coeficiente de determinação do modelo anterior,
os dados foram transformados para uma escala logarítima. Assim, é
possível observar com maior detalhe as diferenças entre as
configurações.

\begin{Shaded}
\begin{Highlighting}[]
\CommentTok{\# Análise Anova log}
\NormalTok{mdl2 }\OtherTok{\textless{}{-}} \FunctionTok{aov}\NormalTok{(Resultado}\SpecialCharTok{\textasciitilde{}}\NormalTok{Configuracao}\SpecialCharTok{+}\NormalTok{Instancia, }\AttributeTok{data =}\NormalTok{ aggdata2)}
\FunctionTok{summary}\NormalTok{(mdl2)}
\end{Highlighting}
\end{Shaded}

\begin{verbatim}
##              Df Sum Sq Mean Sq F value   Pr(>F)    
## Configuracao  1  689.4   689.4  503.21  < 2e-16 ***
## Instancia    33  943.3    28.6   20.87 2.34e-14 ***
## Residuals    33   45.2     1.4                     
## ---
## Signif. codes:  0 '***' 0.001 '**' 0.01 '*' 0.05 '.' 0.1 ' ' 1
\end{verbatim}

\begin{Shaded}
\begin{Highlighting}[]
\FunctionTok{summary.lm}\NormalTok{(mdl2)}\SpecialCharTok{$}\NormalTok{r.squared}
\end{Highlighting}
\end{Shaded}

\begin{verbatim}
## [1] 0.9730564
\end{verbatim}

\begin{Shaded}
\begin{Highlighting}[]
\FunctionTok{par}\NormalTok{(}\AttributeTok{mfrow =} \FunctionTok{c}\NormalTok{(}\DecValTok{2}\NormalTok{, }\DecValTok{2}\NormalTok{))}
\FunctionTok{plot}\NormalTok{(mdl2, }\AttributeTok{pch =} \DecValTok{20}\NormalTok{, }\AttributeTok{las =} \DecValTok{1}\NormalTok{)}
\end{Highlighting}
\end{Shaded}

\includegraphics{estudo_03_files/figure-latex/anova2-1.pdf}

O modelo obtido apresenta bom coeeficiente de determinação. Apesar
disto, o teste estatístico rejeita a hipótese nula ao nível de
significância de \(0.05\). Desta forma, reafirma que existe diferença
entre o desempenho médio das configurações do algoritmo de otimização. O
gráfico da comparação de Dunnet, com intervalo de confiânça de \(95\%\),
é exibido em sequência.

\begin{Shaded}
\begin{Highlighting}[]
\CommentTok{\# Comparação Múltipla 2}

\NormalTok{dtest2 }\OtherTok{\textless{}{-}} \FunctionTok{glht}\NormalTok{(mdl2, }\AttributeTok{linfct =} \FunctionTok{mcp}\NormalTok{(}\AttributeTok{Configuracao =} \StringTok{"Dunnet"}\NormalTok{))}
\NormalTok{dtestIC2 }\OtherTok{\textless{}{-}} \FunctionTok{confint}\NormalTok{(dtest2,}\AttributeTok{level =} \FloatTok{0.95}\NormalTok{)}
\FunctionTok{par}\NormalTok{(}\AttributeTok{mar =} \FunctionTok{c}\NormalTok{(}\DecValTok{5}\NormalTok{, }\DecValTok{8}\NormalTok{, }\DecValTok{4}\NormalTok{, }\DecValTok{2}\NormalTok{), }\AttributeTok{las =} \DecValTok{1}\NormalTok{)}
\FunctionTok{plot}\NormalTok{(dtestIC2,}\AttributeTok{xlim =} \FunctionTok{c}\NormalTok{(}\SpecialCharTok{{-}}\DecValTok{10}\NormalTok{,}\DecValTok{10}\NormalTok{),}\AttributeTok{xlab =} \StringTok{"Diferença das Médias (Escala Logarítmica)"}\NormalTok{)}
\end{Highlighting}
\end{Shaded}

\includegraphics{estudo_03_files/figure-latex/comp2-1.pdf}

No gráfico da diferença é possível verificar que a Configuração 1
apresenta desempenho superior para a classe de problemas em estudo. Além
disso, comprova-se que a diferença de magnitude das respostas médias é
da ordem de \(10^6\).

\hypertarget{conclusuxe3o-e-recomendauxe7uxe3o}{%
\section{Conclusão e
recomendação}\label{conclusuxe3o-e-recomendauxe7uxe3o}}

Com a realização do estudo e como esperado, as configurações do
algoritmo apresentaram diferenças no resultado. Além disso, não
esperava-se que a diferença fosse tão exarcebada, como foi apresentado
nos gráficos. Portanto, vale a pena estudar quais possíveis
configurações podem ser feitas para garantir um bom resultado mais
rápido em determinado tipo de problema.

Com relação ao design do experimento, a utilização do pacote CAISEr
permitiu extrair as informação de planejamento do estudo com melhor
exatidão e menor custo de tempo. Devido ao grande número de dimensões
escolhidas, recomenda-se paralelizar o processo de coleta de dados por
que, quanto maior foi a dimensão do problema, mais tempo levou para
alcançar um resultado.

Enfim, com o experimento foi possível avaliar como as duas configurações
se comportaram e como é importante avaliar as opções de configurações de
heurísticas baseadas em população.

\end{document}
