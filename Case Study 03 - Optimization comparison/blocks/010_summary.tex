% Options for packages loaded elsewhere
\PassOptionsToPackage{unicode}{hyperref}
\PassOptionsToPackage{hyphens}{url}
%
\documentclass[
]{article}
\usepackage{lmodern}
\usepackage{amsmath}
\usepackage{ifxetex,ifluatex}
\ifnum 0\ifxetex 1\fi\ifluatex 1\fi=0 % if pdftex
  \usepackage[T1]{fontenc}
  \usepackage[utf8]{inputenc}
  \usepackage{textcomp} % provide euro and other symbols
  \usepackage{amssymb}
\else % if luatex or xetex
  \usepackage{unicode-math}
  \defaultfontfeatures{Scale=MatchLowercase}
  \defaultfontfeatures[\rmfamily]{Ligatures=TeX,Scale=1}
\fi
% Use upquote if available, for straight quotes in verbatim environments
\IfFileExists{upquote.sty}{\usepackage{upquote}}{}
\IfFileExists{microtype.sty}{% use microtype if available
  \usepackage[]{microtype}
  \UseMicrotypeSet[protrusion]{basicmath} % disable protrusion for tt fonts
}{}
\makeatletter
\@ifundefined{KOMAClassName}{% if non-KOMA class
  \IfFileExists{parskip.sty}{%
    \usepackage{parskip}
  }{% else
    \setlength{\parindent}{0pt}
    \setlength{\parskip}{6pt plus 2pt minus 1pt}}
}{% if KOMA class
  \KOMAoptions{parskip=half}}
\makeatother
\usepackage{xcolor}
\IfFileExists{xurl.sty}{\usepackage{xurl}}{} % add URL line breaks if available
\IfFileExists{bookmark.sty}{\usepackage{bookmark}}{\usepackage{hyperref}}
\hypersetup{
  hidelinks,
  pdfcreator={LaTeX via pandoc}}
\urlstyle{same} % disable monospaced font for URLs
\usepackage[margin=1in]{geometry}
\usepackage{longtable,booktabs}
\usepackage{calc} % for calculating minipage widths
% Correct order of tables after \paragraph or \subparagraph
\usepackage{etoolbox}
\makeatletter
\patchcmd\longtable{\par}{\if@noskipsec\mbox{}\fi\par}{}{}
\makeatother
% Allow footnotes in longtable head/foot
\IfFileExists{footnotehyper.sty}{\usepackage{footnotehyper}}{\usepackage{footnote}}
\makesavenoteenv{longtable}
\usepackage{graphicx}
\makeatletter
\def\maxwidth{\ifdim\Gin@nat@width>\linewidth\linewidth\else\Gin@nat@width\fi}
\def\maxheight{\ifdim\Gin@nat@height>\textheight\textheight\else\Gin@nat@height\fi}
\makeatother
% Scale images if necessary, so that they will not overflow the page
% margins by default, and it is still possible to overwrite the defaults
% using explicit options in \includegraphics[width, height, ...]{}
\setkeys{Gin}{width=\maxwidth,height=\maxheight,keepaspectratio}
% Set default figure placement to htbp
\makeatletter
\def\fps@figure{htbp}
\makeatother
\setlength{\emergencystretch}{3em} % prevent overfull lines
\providecommand{\tightlist}{%
  \setlength{\itemsep}{0pt}\setlength{\parskip}{0pt}}
\setcounter{secnumdepth}{-\maxdimen} % remove section numbering
\ifluatex
  \usepackage{selnolig}  % disable illegal ligatures
\fi

\author{}
\date{\vspace{-2.5em}}

\begin{document}

A fim de solucionar problemas de otimização, heurísticas são utilizadas
a fim de encontrar uma solução viável em tempo hábil, dado que
algoritmos para encontrar a solução ótima exata levariam anos para obter
sucesso. Uma dessas heurísticas são algoritmos baseados em população,
que utilizam um conjunto de soluções-candidatas e, com uma série de
operações de variação e seleção nelas, exploram o espaço amostral do
problema em busca do ótimo. Neste ensaio, utiliza-se o método de
evolução diferencial.

Este trabalho explora como duas diferentes configurações
(\protect\hyperlink{tabel1}{tabela 1}) do método citado impactam no
resultado, observando qual destas obtém melhor desempenho. Para testar
as configurações, utiliza-se uma função de Rosenbrock, com o intervalo
de dimensões de interesse de \([2,150]\).

\begin{longtable}[]{@{}lll@{}}
\toprule
Configuração & Mutação & Recombinação\tabularnewline
\midrule
\endhead
1 & Aleatória (f = 4) & Aritmética\tabularnewline
2 & Melhores (f = 3) & Binária (cr = 0.7)\tabularnewline
\bottomrule
\end{longtable}

As próximas etapas deste estudo de caso é apresentado a seguir:

\begin{enumerate}
\def\labelenumi{\arabic{enumi}.}
\tightlist
\item
  Formulação das hipóteses de teste;
\item
  Cálculo dos tamanhos amostrais;
\item
  Coleta e tabulação dos dados;
\item
  Teste das hipóteses;
\item
  Estimação da magnitude da diferença entre os métodos;
\item
  Verificação das premissas dos testes;
\item
  Conclusões
\item
  Discussão sobre os processos do estudo
\end{enumerate}

\end{document}
